\documentclass{report}

\usepackage{graphicx}
\usepackage{booktabs}
\usepackage{placeins}
\usepackage{subcaption}
\usepackage{float}


\title{GEO1001 Homework 01 Report}
\author{Noortje van der Horst}
\date{21 September 2020}

\begin{document}
	
	\maketitle
	
	\section{Lesson A1}
	For this assignment you will use data collected from 5 heat stress sensors placed somewhere in the Netherlands during this summer. The sensors are Kestrel 5400 and their specs are included within the assignment materials. In order to identify if the dataset is of any value to your “employer”, it is your job to deeply analyse the dataset and derive hypothesis from it. The work you need to do for this assignment can roughly be subdivided in 4 tasks related to each independent lesson.
	
	
	\subsection{Question 1.1}
	\textbf{Compute mean statistics (mean, variance and standard deviation for each of the sensors variables), what do you observe from the results?}
	
	\FloatBarrier
	% Table generated by Excel2LaTeX from sheet 'means'
	\begin{table}[htbp]
		\centering
		\caption{Means all sensors}
		\resizebox{\textwidth}{!}{\begin{tabular}{llllll}
			\textbf{Sensors} & A     & B     & C     & D     & E \\
			\textbf{Direction ‚ True} & 209.41 & 183.41 & 183.59 & 198.33 & 223.96 \\
			\textbf{Wind Speed} & 1.29  & 1.24  & 1.37  & 1.58  & 0.6 \\
			\textbf{Crosswind Speed} & 0.96  & 0.84  & 0.96  & 1.21  & 0.44 \\
			\textbf{Headwind Speed} & 0.16  & -0.13 & -0.26 & -0.3  & 0.19 \\
			\textbf{Temperature} & 17.97 & 18.07 & 17.91 & 18.0  & 18.35 \\
			\textbf{Globe Temperature} & 21.54 & 21.8  & 21.59 & 21.36 & 21.18 \\
			\textbf{Wind Chill} & 17.84 & 17.95 & 17.77 & 17.84 & 18.29 \\
			\textbf{Relative Humidity} & 78.18 & 77.88 & 77.96 & 77.94 & 76.79 \\
			\textbf{Heat Stress Index} & 17.9  & 18.0  & 17.83 & 17.92 & 18.29 \\
			\textbf{Dew Point} & 13.55 & 13.53 & 13.46 & 13.51 & 13.56 \\
			\textbf{Psychro Wet Bulb Temperature} & 15.27 & 15.3  & 15.2  & 15.26 & 15.41 \\
			\textbf{Station Pressure} & 1016.17 & 1016.66 & 1016.69 & 1016.73 & 1016.17 \\
			\textbf{Barometric Pressure} & 1016.13 & 1016.62 & 1016.65 & 1016.69 & 1016.13 \\
			\textbf{Altitude} & -25.99 & -30.06 & -30.34 & -30.65 & -25.96 \\
			\textbf{Density Altitude} & 137.32 & 135.58 & 129.62 & 132.41 & 150.84 \\
			\textbf{NA Wet Bulb Temperature} & 15.98 & 16.0  & 15.93 & 15.92 & 15.94 \\
			\textbf{WBGT} & 17.25 & 17.32 & 17.23 & 17.18 & 17.19 \\
			\textbf{TWL} & 301.39 & 299.45 & 301.9 & 305.25 & 284.12 \\
			\textbf{Direction ‚ Mag} & 208.91 & 183.22 & 183.08 & 197.83 & 223.9 \\
		\end{tabular}}%
		\label{tab:means}%
	\end{table}%
	\FloatBarrier


	\FloatBarrier
	% Table generated by Excel2LaTeX from sheet 'var'
	\begin{table}[htbp]
		\centering
		\caption{Variances all sensors}
		\resizebox{\textwidth}{!}{\begin{tabular}{llllll}
			\textbf{Sensors} & A     & B     & C     & D     & E \\
			\textbf{Direction ‚ True} & 10108.94 & 9977.22 & 7703.36 & 8133.89 & 9308.29 \\
			\textbf{Wind Speed} & 1.25  & 1.3   & 1.43  & 1.74  & 0.51 \\
			\textbf{Crosswind Speed} & 0.93  & 0.88  & 1.04  & 1.45  & 0.32 \\
			\textbf{Headwind Speed} & 1.03  & 1.26  & 1.27  & 1.23  & 0.32 \\
			\textbf{Temperature} & 15.86 & 16.63 & 16.1  & 16.11 & 19.04 \\
			\textbf{Globe Temperature} & 68.19 & 66.05 & 67.94 & 61.2  & 63.22 \\
			\textbf{Wind Chill} & 16.26 & 17.04 & 16.54 & 16.56 & 19.14 \\
			\textbf{Relative Humidity} & 376.01 & 408.62 & 374.62 & 389.86 & 406.49 \\
			\textbf{Heat Stress Index} & 15.0  & 15.44 & 15.36 & 15.12 & 18.48 \\
			\textbf{Dew Point} & 9.72  & 9.64  & 10.08 & 10.07 & 9.42 \\
			\textbf{Psychro Wet Bulb Temperature} & 6.94  & 6.77  & 7.24  & 7.04  & 7.0 \\
			\textbf{Station Pressure} & 38.47 & 36.84 & 37.69 & 34.99 & 38.94 \\
			\textbf{Barometric Pressure} & 38.47 & 36.83 & 37.68 & 34.95 & 38.94 \\
			\textbf{Altitude} & 2663.64 & 2545.71 & 2608.53 & 2419.72 & 2692.35 \\
			\textbf{Density Altitude} & 26510.04 & 26863.31 & 26986.6 & 26516.13 & 29714.93 \\
			\textbf{NA Wet Bulb Temperature} & 10.01 & 9.81  & 10.48 & 9.99  & 9.43 \\
			\textbf{WBGT} & 16.14 & 15.84 & 16.55 & 15.51 & 15.49 \\
			\textbf{TWL} & 814.77 & 790.07 & 766.53 & 616.01 & 1289.91 \\
			\textbf{Direction ‚ Mag} & 10105.68 & 9975.45 & 7704.62 & 8135.32 & 9268.01 \\
		\end{tabular}}%
		\label{tab:vars}%
	\end{table}%
	\FloatBarrier


	\FloatBarrier
	% Table generated by Excel2LaTeX from sheet 'std'
	\begin{table}[htbp]
		\centering
		\caption{Standard deviations all sensors}
		\resizebox{\textwidth}{!}{\begin{tabular}{llllll}
			\textbf{Sensor} & A     & B     & C     & D     & E \\
			\textbf{Direction ‚ True} & 100.54 & 99.89 & 87.77 & 90.19 & 96.48 \\
			\textbf{Wind Speed} & 1.12  & 1.14  & 1.2   & 1.32  & 0.72 \\
			\textbf{Crosswind Speed} & 0.96  & 0.94  & 1.02  & 1.2   & 0.56 \\
			\textbf{Headwind Speed} & 1.02  & 1.12  & 1.13  & 1.11  & 0.56 \\
			\textbf{Temperature} & 3.98  & 4.08  & 4.01  & 4.01  & 4.36 \\
			\textbf{Globe Temperature} & 8.26  & 8.13  & 8.24  & 7.82  & 7.95 \\
			\textbf{Wind Chill} & 4.03  & 4.13  & 4.07  & 4.07  & 4.37 \\
			\textbf{Relative Humidity} & 19.39 & 20.21 & 19.36 & 19.74 & 20.16 \\
			\textbf{Heat Stress Index} & 3.87  & 3.93  & 3.92  & 3.89  & 4.3 \\
			\textbf{Dew Point} & 3.12  & 3.1   & 3.18  & 3.17  & 3.07 \\
			\textbf{Psychro Wet Bulb Temperature} & 2.64  & 2.6   & 2.69  & 2.65  & 2.65 \\
			\textbf{Station Pressure} & 6.2   & 6.07  & 6.14  & 5.92  & 6.24 \\
			\textbf{Barometric Pressure} & 6.2   & 6.07  & 6.14  & 5.91  & 6.24 \\
			\textbf{Altitude} & 51.61 & 50.46 & 51.07 & 49.19 & 51.89 \\
			\textbf{Density Altitude} & 162.82 & 163.9 & 164.28 & 162.84 & 172.38 \\
			\textbf{NA Wet Bulb Temperature} & 3.16  & 3.13  & 3.24  & 3.16  & 3.07 \\
			\textbf{WBGT} & 4.02  & 3.98  & 4.07  & 3.94  & 3.94 \\
			\textbf{TWL} & 28.54 & 28.11 & 27.69 & 24.82 & 35.92 \\
			\textbf{Direction ‚ Mag} & 100.53 & 99.88 & 87.78 & 90.2  & 96.27 \\
		\end{tabular}}%
		\label{tab:std}%
	\end{table}%
	\FloatBarrier
		
	All tables above have been made by exporting the generated mean statistics to csv files, importing them in excel, and lastly converting these excel tables to LaTeX tables with a plugin. The excelsheet used has been provided with this report. I had trouble with pandas recognizing the datetimes, so mean statistics for the dates and times have unfortunately not been included.
	
	From these tables, it seems that all sensors measured values that are (roughtly) comparable. The sensor's locations will probably not have differed radically from each other. Sensor E measured noticably less wind and higher temperatures than the other sensors. The variance for sensor E is also larger for several variables, which could explain some of the difference in means from the other sensors. The standard deviations show less of a difference. Larger differences in means do seem to coincide with larger differences in spread and standard deviation.
	
	
	\subsection{Question 1.2}
	\textbf{Create 1 plot that contains histograms for the 5 sensors Temperature values. Compare histograms with 5 and 50 bins, why is the number of bins important?}

	\begin{figure}[h!]
		\centering
		\begin{subfigure}[b]{0.45\linewidth}
			\includegraphics[width=\linewidth]{GEO1001_hw01_images/GEO1001_hw01_A1_hist_5.png}
			\caption{Historgam 5 bins}
		\end{subfigure}
		\begin{subfigure}[b]{0.45\linewidth}
			\includegraphics[width=\linewidth]{GEO1001_hw01_images/GEO1001_hw01_A1_hist_50.png}
			\caption{Histogram 50 bins}
		\end{subfigure}
		\caption{Temperature historgam, different nr. of bins}
		\label{fig:histA1}
	\end{figure}

	Figure \ref{fig:histA1} shows the importance of choosing the right number of bins when making a histogram. Both historgams show the distribution of measured temperatures per sensor, but figre b contains a lot more information about the distribution than figure a. A clear peak is indicated, as well as some information about the tails of the distribution and its skewdness. However, figure a does show the differences in distribution between the 5 sensors more clearly. Figure b could be imporved to be less "busy" (e.g. displaying 1 histogram per sensor next to eachother), but as it is right now it is hard to see clear distinctions bewteen sensors.
	
	\subsection{Question 1.3}
	\textbf{Create 1 plot where frequency poligons for the 5 sensors Temperature values overlap in different colors with a legend.}
	
	\begin{figure}[h!]
		\includegraphics[width=\linewidth]{GEO1001_hw01_images/GEO1001_hw01_A1_polygons.png}
		\caption{Frequency polygons per sensor, Temperature}
		\label{fig:freqpols}
	\end{figure}

	From Firgue \ref{fig:freqpols} it is again visible that the Temperature meansurements for the 5 sensors follow roughtly the same distribution. Sensor E measured higher temperatures, visible in a shorter "left tail" and longer "right tail". All sensors peaks lie around 17 degrees Celsius, this is also where their means will roughly be located.
	
	\subsection{Question 1.4}
	\textbf{Generate 3 plots that include the 5 sensors boxplot for: Wind Speed, Wind Direction and Temperature.}
	
	
	\begin{figure}[H]
		\centering
		\begin{subfigure}[b]{0.7\linewidth}
			\includegraphics[width=\linewidth]{GEO1001_hw01_images/GEO1001_hw01_A1_box_wind.png}
			\caption{Boxplots 5 sensors, Wind Speed}
			\label{fig:boxwind}
		\end{subfigure}
	
		\begin{subfigure}[b]{0.7\linewidth}
			\includegraphics[width=\linewidth]{GEO1001_hw01_images/GEO1001_hw01_A1_box_direction.png}
			\caption{Boxplots 5 sensors, Wind Direction}
			\label{fig:boxdir}
		\end{subfigure}
		
		\begin{subfigure}[b]{0.7\linewidth}
			\includegraphics[width=\linewidth]{GEO1001_hw01_images/GEO1001_hw01_A1_box_temp.png}
			\caption{Boxplots 5 sensors, Temperature}
			\label{fig:boxtemp}
		\end{subfigure}
	\end{figure}
	
	
	\section{Lesson A2}
	
	\subsection{Question 2.1}
	\textbf{Plot PMF, PDF and CDF for the 5 sensors Temperature values. Describe the behaviour of the distributions, are they all similar? what about their tails?}
	
	\begin{figure}[H]
		\centering
		\begin{subfigure}[b]{0.85\linewidth}
			\includegraphics[width=\linewidth]{GEO1001_hw01_images/GEO1001_hw01_A2_PMF.png}
			\caption{Probability mass function, Temperature, all sensors}
			\label{fig:pmf}
		\end{subfigure}
		
		\begin{subfigure}[b]{0.85\linewidth}
			\includegraphics[width=\linewidth]{GEO1001_hw01_images/GEO1001_hw01_A2_PDF.png}
			\caption{Probability density function, Temperature, all sensors}
			\label{fig:pdf}
		\end{subfigure}
		
		\begin{subfigure}[b]{0.85\linewidth}
			\includegraphics[width=\linewidth]{GEO1001_hw01_images/GEO1001_hw01_A2_CDF.png}
			\caption{Cumulative density function, Temperature, all sensors}
			\label{fig:cdf}
		\end{subfigure}
	\end{figure}
	
	\subsection{Question 2.2}
	\textbf{For the Wind Speed values, plot the pdf and the kernel density estimation. Comment the differences.}
	
	
	\begin{figure}[h!]
		\centering
		\includegraphics[width=\linewidth]{GEO1001_hw01_images/GEO1001_hw01_A2_kde_compare.png}
		\caption{PDF and kde for all sensors, Temperature}
		\label{fig:kde}
	\end{figure}

	\begin{figure}[h!]
		\centering
		\includegraphics[width=\linewidth]{GEO1001_hw01_images/GEO1001_hw01_A2_kde_original.png}
		\caption{PDF and kde for all sensors, Temperature, unequal scale}
		\label{fig:kdeunequal}
	\end{figure}
	
	\section{Lesson A3}
	
	\subsection{Question 3.1}
	\textbf{Compute the correlations between all the sensors for the variables: Temperature, Wet Bulb Globe, Crosswind Speed. Use Pearson’s and Spearmann’s rank coefficients. Make a scatter plot with both coefficients with the 3 variables.}
	
	\begin{figure}[H]
		\centering
		\includegraphics[width=\linewidth]{GEO1001_hw01_images/GEO1001_hw01_A3_correlations.png}
		\caption{correlations between all sensors, Spearman \& Pearson}
		\label{fig:corr}
	\end{figure}
	
	
	\subsection{Question 3.2}
	\textbf{What can you say about the sensors’ correlations?}
	
	Figure \ref{fig:corr} shows sensor E is significantly less correlated with the other sensors: it's Pearson's r and Spearman's $\rho$ are lower for all sensor combinations with sensor E. All Pearson correlation coefficients for Temperature and Wet Bulb Globe Temperature are close to 1, which means there's a very strong positive (linear) correlation between the sensors. The Spearman's rank correlation coefficients are also very close to 1 for these variables. The Pearson's r for Crosswind Speed is significantly lower, lying around 0.5, meaning there's a less strong, but still measurable, correlation between the sensors for this variable. The Spearman's $\rho$ is a bit higher, around 0.6, meaning the colleraltion for Crosswind Speed is not (completely) linear.
	
	\subsection{Question 3.3}
	\textbf{If we told you that that the sensors are located as follows, hypothesize which location would you assign to each sensor and reason your hypothesis using the correlations.}
	
	From Figure \ref{fig:corr}, it is immediately clear that sensor E correlates significantly less with the other sensors. It is therefor likely this sensor is located in the most different spot: the secluded spot at the top. The rest of the locations are less easy to decipher. Differences bewteen correlations are very small. Possible differences could include direction of open spaces around the sensor, terrain type below the sensor, and cover provided by nearby structures. These characteristics are listed in Table \ref{tab:locations}. Locations 1 to 4 are numbered according to Figure \ref{fig:locations}, excluding the location likely to be sensor E. Directions were determined using the top of the image as an (arbitary) north. 
	
	\FloatBarrier
	% Table generated by Excel2LaTeX from sheet 'CI'
	\begin{table}[htbp]
		\centering
		\caption{Possible causes of differences between sensors}
		\resizebox{\textwidth}{!}{\begin{tabular}{lllll}
				\textbf{Location} & \textbf{Dir. open spaces} & \textbf{Dir. cover} & \textbf{Terrain type} & \textbf{Likely sensor} \\
				1     & N   & E, S, W  & field  &C  \\
				2     & N, NE   & E, S, W  & tiles  &D  \\
				3     & N, E  & S, W  & field  &A  \\
				4     & N, S   & E, W  & grassy field  &B  \\	
		\end{tabular}}%
		\label{tab:locations}%
	\end{table}%
	\FloatBarrier
	
	Looking closely at the correlations, it would seem sensor D correlates the least with the other sensors (excluding E).This could be the sensor located on the pavement, which would account for higher measured Temperature and Wet Bulb Globe values. For Crosswind Speed, there seem to be two pairs of sensors which relatively correlate more: A \& B and C \& D. These pairs could be combined with the pairs 1 \& 2 and 3 \& 4 of the sensor locations, which would have the most cmoparable wind speed value pairs. Since C is paired with D, location 1 could correspond to it. The choice of where to place A and B was based on the fact that A correlates a slight bit more with C then B does with C. Location 1 and 3 have the same type of grass, while location 4 has a more lush terrain. This could be a factor in the difference in correlation of sensor A and B with the other sensors. These locations are based on hypotheses, they are not decided with absolute certainty.
	
	The result of the analysis is displayed in Figure \ref{fig:locations}.
	
	\begin{figure}[H]
		\includegraphics[width=\linewidth]{GEO1001_hw01_images/SensorsSketch_guess.png}
		\caption{Hypothesized locations of sensors}
		\label{fig:locations}
	\end{figure}
	
	
	\section{Lesson A4}
	
	\subsection{Question 4.1}
	\textbf{Plot the CDF for all the sensors and all the variables, then compute the 95\% confidence intervals for all the variables and sensors and save them in a table (txt or csv form).}
	
	Table \ref{tab:ci} shows the confidence intervals calculated from the plotted CDFs (Figure \ref{fig:a4cdf}). The temperature values were determined by finding the x-value for y-values 0.025 and 0.975 on the CDF, since these would correspond to the lower and upper 2,5\% of the symmetrical CDF, leaving 95\% of the values. The means for each sensor are also added to Table \ref{tab:ci}. The confidence intervals do not lie perfectly symmetrical around these means, as one might expect. This is because the interval was extrapolated from a real-life sample, instead of a theoretical distribution function (like a normal distribution). Looking at the shape of the CDF of the Wind Speeds, it would seem likely this variable follows an exponential distribution (as opposed to the (expected) normal distribution of the Temperature).
	
	\begin{figure}[H]
		\includegraphics[width=\linewidth]{GEO1001_hw01_images/GEO1001_hw01_A4_CDF.png}
		\caption{Visualisation CDF of all sensors, Temperature \& Wind Speed}
		\label{fig:a4cdf}
	\end{figure}

	\FloatBarrier
	% Table generated by Excel2LaTeX from sheet 'CI'
	\begin{table}[htbp]
		\centering
		\caption{Confidence Interval with mean, Temperature \& Wind Speed}
		\resizebox{\textwidth}{!}{\begin{tabular}{lllllll}
			\textbf{Sensor} & \textbf{CI min Temperature} & \textbf{mean Temperature} & \textbf{CI max Temperature} & \textbf{CI min Wind Speed} & \textbf{mean Wind Speed} & \textbf{CI max Wind Speed} \\
			A     & 9.6   & 18.0  & 28.1  & 0.0   & 1.3   & 4.1 \\
			B     & 9.8   & 18.1  & 28.4  & 0.0   & 1.2   & 3.9 \\
			C     & 9.3   & 17.9  & 28.2  & 0.0   & 1.4   & 4.2 \\
			D     & 9.6   & 18.0  & 27.9  & 0.0   & 1.6   & 4.7 \\
			E     & 10.6  & 18.4  & 30.5  & 0.0   & 0.6   & 2.4 \\
		\end{tabular}}%
		\label{tab:ci}%
	\end{table}%
	\FloatBarrier
	
	
	\subsection{Question 4.2}
	\textbf{Test the hypothesis: the time series for Temperature and Wind Speed are the same for sensors:
	\begin{itemize}
		\item 1) E, D;
		\item 2) D, C;
		\item 3) C, B;
		\item 4) B, A.
	\end{itemize}	}
	
	\FloatBarrier
	% Table generated by Excel2LaTeX from sheet 'Sheet8'
	\begin{table}[htbp]
		\centering
		\caption{t and p values for tested hypotheses}
		\resizebox{\textwidth}{!}{\begin{tabular}{lllll}
			\textbf{sensors} & \textbf{Temperature t} & \textbf{Temperature p} & \textbf{Wind Speed t} & \textbf{Wind Speed p} \\
			ED    & 2.9985 & 0.0027 & -32.6596 & 0.0 \\
			DC    & 0.7294 & 0.4658 & 5.8712 & 0.0 \\
			CB    & -1.3238 & 0.1856 & 3.9088 & 0.0001 \\
			BA    & 0.8408 & 0.4005 & -1.5006 & 0.1335 \\
		\end{tabular}}%
		\label{tab:ttest}%
	\end{table}%
	\FloatBarrier
	

	\subsection{Question 4.3}
	\textbf{What could you conclude from the p-values?}
	
	The hypothesis testing was done with the following parameters:
	\begin{itemize}
		\item H0: $\mu1 - \mu2 = 0$
		\item H1: $\mu1 - \mu2 > 0$
		\item $\alpha = 0.05$
	\end{itemize}
	If the null hypothesis holds, the time series from both sensors would likely be equal. The p-value resulting from the t-tests done for the previous question describes the probability that the observed difference between the means of the 2 samples is due to chance. If this probability is not significant, i.e. smaller than $\alpha$, it is not likely the difference was caused by chance. This means it is likely the samples are actually different, with 1 or more factors other than chance causing a difference in means. The difference in means for Temperature between sensor D and C for example, is 47\% likely to have been due to chance. Here, the H0 was accepted, since this is significantly larger than the predefined 5\%. The conclusions drawn from the p-values are summarized in table \ref{tab:ttestconc}.
	
	\FloatBarrier
	% Table generated by Excel2LaTeX from sheet 'Sheet8'
	\begin{table}[h!]
		\centering
		\caption{conclusions summarized}
		\resizebox{\textwidth}{!}{\begin{tabular}{lllll}
				\textbf{sensors} & \textbf{Temperature} & \textbf{conclusion} & \textbf{Wind Speed} & \textbf{conclusion} \\
				ED    & 0.0027 $<$ 0.05 & reject H0 & 0.0000 $<$ 0.05 & reject H0 \\
				DC    & 0.4658 $>$ 0.05 & accept H0 & 0.0000 $<$ 0.05 & reject H0 \\
				CB    & 0.1856 $>$ 0.05 & reject H0 & 0.0001 $<$ 0.05 & reject H0 \\
				BA    & 0.4005 $>$ 0.05 & accept H0 & 0.1335 $>$ 0.05 & accept H0 \\
		\end{tabular}}%
		\label{tab:ttestconc}%
	\end{table}%
	\FloatBarrier


	
\end{document}

	